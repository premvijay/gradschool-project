\documentclass[12pt]{article}
\usepackage{amsmath,amssymb}
\usepackage{hyperref}
\usepackage[margin=1in]{geometry}

\renewcommand\thesection{}
\renewcommand\thesubsection{\arabic{subsection}}
\renewcommand\thesubsubsection{\thesubsection\alph{subsubsection}}


\begin{document}
	



\section{15th June 2020}

\subsection{Basic reading}
Dodelson ch-7. Inhomogeneities : reproduce the plot in figure 7.11 using the BBKS transfer function - read the whole chapter and reproduce other plots.

\subsection{Advanced reading}
Read the introduction of \url{https://arxiv.org/pdf/1706.09906.pdf} and summarize it.

\subsection{Try N-body sims}
Do a particle mesh code following the tutorial by Andrey Kravstov \url{https://astro.uchicago.edu/~andrey/Talks/PM/pm.pdf} 


\section{22nd June 2020}

\begin{itemize}
\item Do $\sigma_8$ normalisation of BBKS power spectrum and generate gaussian random field with that power spectrum.
\item  Visualise the generated gaussian random field in physical space and compare with the power law one.
\item Read more about halo bias.
\end{itemize}
 
 
\section{29th June 2020}
\begin{itemize}
\item Similar to the smoothed variance $\sigma_8$, derive smoothed correlation function with different arbitrary smoothing.
\item Take of particle positions in {snapshot\_200*} of simulation scm1024 and use CIC code to get density field. Visualise and compute power spectrum.
\item Smooth it and then visualize and compute power spectrum.
\item Read more about halos from Cooray and Sheth section 3.
\end{itemize}
 

\newpage



\section*{Correlation function power spectrum relation}

\begin{align}
P(\vec{k}) &= \int \xi(\vec{r}) ~e^{i \vec{k} \dot \vec{r}} ~d^3r\\
\xi(\vec{r}) &= \frac{1}{(2\pi)^3} \int P(\vec{k}) ~e^{-i \vec{k} \dot \vec{r}} ~d^3 k\\
\xi(r) &= \frac{4 \pi}{(2\pi)^3} \int_{0}^{\infty} P(k) ~k^2 ~\frac{\sin(kr)}{kr} dk\\
\xi(r) &= \frac{1}{2\pi^2} \int_{-\infty}^{\infty} P(k) ~k^3 ~\frac{\sin(kr)}{kr} d(\ln k)\\
\xi(r) &= \int_{-\infty}^{\infty} \Delta^2(k) ~\frac{\sin(kr)}{kr} d(\ln k)
\end{align}


\section{$\sigma_8$ normalisation}
It is defined as the root mean square of density variation after smoothing by correlating with spherical tophat function of radius 8 h$^{-1}$ Mpc. In fourier space, this is equivalent to multiplying by the fourier transform of that tophat .
\begin{align}
W_s(k) &= 3 ~\frac{j_1(k R_8)}{k R_8} \quad \text{where } R_8 = 8 \text{ h$^{-1}$ Mpc}
\end{align}

\begin{align}
\sigma_8 &= \sqrt{ \frac{1}{(2 \pi)^3} \int W_s^2(k) ~|\delta (\vec{k})|^2 ~d^3k}\\
\sigma_8^2 &=  \frac{1}{(2 \pi)^3} \int W_s^2(k) ~|\delta (\vec{k})|^2 ~d^3k\\
&= \frac{1}{(2 \pi)^3} \int W_s^2(k) ~|\delta (\vec{k})|^2 ~k^2 dk ~d\Omega\\
&= \frac{1}{(2 \pi)^3} \int W_s^2(k) ~k^2 dk \int |\delta (\vec{k})|^2  ~d\Omega\\
&= \frac{1}{(2 \pi)^3} \int W_s^2(k) ~k^2 ~4\pi P(k) ~dk\\
&= \frac{1}{2 \pi^2} \int W_s^2(k) ~k^2 ~P(k) ~dk\\
&= \int W_s^2(k) ~\Delta^2(k) ~d(\ln k)
\end{align}

That integrand drops away from the that 8 h$^{-1}$ Mpc scale

\section{Correlation after smoothing:}

\begin{align}
\xi_{s_1,s_2} (r) = \int_{-\infty}^{\infty} \Delta^2(k) ~W_{s_1}(k) ~W_{s_2}(k) ~\frac{\sin(kr)}{kr} d(\ln k)
\end{align}

\newpage

\section{Large Scale Structure notes}

Inhomonenous evolution can't be done completely analytically, eventhough it can be simulated. Analytical tools/models are important to gain deeper understanding.  Simulations help in making, testing and refining these analytical tools along with the observations.\\

\subsection{FLRW background evolution}

\subsection{Newtonian equations for imhomogeneous CDM}

\subsection{Growth of Structure}

\subsubsection{Linear solutions to inhomogeneous CDM}
\subsubsection{Eulerian - 2nd order perturbation theory}
\subsubsection{Lagrangian approach - Zel'dovich approximations}
\subsubsection{Spherical collapse}

\subsection{Halo Model}
\subsubsection{Halo Bias}
Split $\delta(\vec{x})$ into small scale and large scale components. 

\subsection*{Halo assembly bias}

\newpage

























\end{document}